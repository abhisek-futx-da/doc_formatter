\documentclass{ieeetj}
\usepackage{cite}
\usepackage{amsmath,amssymb,amsfonts}
\usepackage{algorithmic}
\usepackage{graphicx,color}
\usepackage{textcomp}
\usepackage{xcolor}
\usepackage{hyperref}
\hypersetup{hidelinks=true}
\usepackage{algorithm,algorithmic}
\def\BibTeX{{\rm B\kern-.05em{\sc i\kern-.025em b}\kern-.08em
    T\kern-.1667em\lower.7ex\hbox{E}\kern-.125emX}}
\AtBeginDocument{\definecolor{tmlcncolor}{cmyk}{0.93,0.59,0.15,0.02}\definecolor{NavyBlue}{RGB}{0,86,125}}




\def\OJlogo{\vspace{-4pt}$<$Society logo(s) and publication title will appear here.$>$}
\def\seclogo{\vspace{10pt}$<$Society logo(s) and publication title will appear here.$>$}

\def\authorrefmark#1{\ensuremath{^{\textbf{#1}}}}
\usepackage{graphicx}
\usepackage{booktabs}

\begin{document}

A Novel Spatio-Topological Embeddings for Efficient \& Model-free
Redundant Node Placement in 6G IoT Networks

\hypertarget{abstract-the-development-of-6g-has-accelerated-the-usage-of-iot-for-data-collection.-with-the-continuous-usage-and-ubiquitous-connectivity-the-nodes-batteries-depletes-soon-and-create-the-coverage-hole-in-the-network-imposing-the-connectivity-challenges-in-6g-iot.-the-problem-has-been-dealt-with-a-novel-set-of-spatial-features-with-topological-embeddings-extraction-using-graphical-convolutional-network.-the-deep-deterministic-policy-gradient-in-the-continuous-action-space-trains-the-agent-for-the-optimal-placement-of-redundant-nodes.-the-complete-methodology-with-spatio-topological-features-have-seen-an-improvement-upto-13.1-in-energy-residual-and-19.4-in-the-uniform-load-distribution-than-the-state-of-the-art-methods-with-stable-network-connectivity.-also-the-analysis-at-various-environmental-conditions-with-varying-holes-density-and-sensors-density.-the-proposed-scheme-has-shown-the-improvement-under-adverse-coverage-conditions-too.}{%
\section{\texorpdfstring{\textbf{Abstract:} The development of 6G has
accelerated the usage of IoT for data collection. With the continuous
usage and ubiquitous connectivity, the nodes' batteries depletes soon
and create the coverage hole in the network, imposing the connectivity
challenges in 6G IoT. The problem has been dealt with a novel set of
spatial features with topological embeddings extraction using graphical
convolutional network. The deep deterministic policy gradient in the
continuous action space trains the agent for the optimal placement of
redundant nodes. The complete methodology with spatio-topological
features have seen an improvement upto 13.1\% in energy residual and
19.4\% in the uniform load distribution than the state-of-the-art
methods with stable network connectivity. Also the analysis at various
environmental conditions with varying holes' density and sensors'
density. The proposed scheme has shown the improvement under adverse
coverage conditions
too.}{Abstract: The development of 6G has accelerated the usage of IoT for data collection. With the continuous usage and ubiquitous connectivity, the nodes' batteries depletes soon and create the coverage hole in the network, imposing the connectivity challenges in 6G IoT. The problem has been dealt with a novel set of spatial features with topological embeddings extraction using graphical convolutional network. The deep deterministic policy gradient in the continuous action space trains the agent for the optimal placement of redundant nodes. The complete methodology with spatio-topological features have seen an improvement upto 13.1\% in energy residual and 19.4\% in the uniform load distribution than the state-of-the-art methods with stable network connectivity. Also the analysis at various environmental conditions with varying holes' density and sensors' density. The proposed scheme has shown the improvement under adverse coverage conditions too.}}\label{abstract-the-development-of-6g-has-accelerated-the-usage-of-iot-for-data-collection.-with-the-continuous-usage-and-ubiquitous-connectivity-the-nodes-batteries-depletes-soon-and-create-the-coverage-hole-in-the-network-imposing-the-connectivity-challenges-in-6g-iot.-the-problem-has-been-dealt-with-a-novel-set-of-spatial-features-with-topological-embeddings-extraction-using-graphical-convolutional-network.-the-deep-deterministic-policy-gradient-in-the-continuous-action-space-trains-the-agent-for-the-optimal-placement-of-redundant-nodes.-the-complete-methodology-with-spatio-topological-features-have-seen-an-improvement-upto-13.1-in-energy-residual-and-19.4-in-the-uniform-load-distribution-than-the-state-of-the-art-methods-with-stable-network-connectivity.-also-the-analysis-at-various-environmental-conditions-with-varying-holes-density-and-sensors-density.-the-proposed-scheme-has-shown-the-improvement-under-adverse-coverage-conditions-too.}}

\textbf{Keywords: 6g IoT, topological features, GCN, DDPG, Reinforcement
Learning}

\hypertarget{introduction}{%
\section{Introduction}\label{introduction}}

Moving towards a ubiquitous network with every device speaking with
other, 6G era has gained tremendous growth. One important expectation in
the 6G era is that machines and things will be the main consumers of
mobile data traffic, and thus there will be more than 55 billion devices
connected to the Internet by the end of 2025. These devices will
continuously sense, process, act, and communicate with the surrounding
environment, generating more than 73 zettabytes of data per year
{[}1{]}. With the advancement of communication and AI, the 6G is
supposed to handle massive communication, hyper-large AI data handling,
reliable and low-latency communication, and ubiquitous connectivity
{[}2{]}. The primary challenge and the basic requirement in this new era
are continuous connectivity with devices like self driving cars, drones,
IoT, robot operated factories etc. Traditional IoT applications have
been advanced with the 6G communication capabilities which lead to more
processing at IoT nodes using AI. The more the processing is done at
local or edge devices or cloud, more is the battery consumption. The
massive transmission in 6G imposes a challenge of continuous
connectivity and longer battery life. This creates coverage holes in the
network {[}3{]}. On the other side, the selfish behaviour of a few nodes
also leads to connectivity issues. These coverage holes pose the
challenge of connectivity and quality of service in the 6G powered
intelligent IoT services, so the solution to the criticality of the
coverage holes is the primary motive of this research article.

Many researchers have focused on the coverage hole issues. It involves
the study into two phases: hole detection and hole recovery. The
detection of holes is primarily focused on Voronoi diagram and Delaunay
triangulation, which requires the exact geo locations of the nodes,
which is impractical and expensive for the large applications of IoT in
6G {[}4{]}. Another topologically based hole detection is distance-based
schemes {[}5{]}{[}6{]}{[}7{]}, which also require the precise
coordinates of the sensor nodes, which is again impractical and
expensive in case of mobile IoT nodes. Another category of research in
the hole detection is the confident information coverage and disk
coverage model. The disk coverage model is too idealistic for 6G IoT as
shown in simulated figure 1, whereas confident information coverage
defines the coverage from the estimation perspective and is suitable for
6G IoT applications {[}3{]}{[}8{]}. Another approach considering the
estimation perspective in the hole detection and estimating the boundary
nodes to detect the hole and its irregular shape is using the improved
Gaussian mixture model {[}4{]}. This present work considers our previous
work on hole detection {[}4{]} and primarily focuses on the hole
recovery mechanism.

\includegraphics[width=0.9\textwidth]{images/image1.png}

Figure 1: Coverage holes' detection with disk coverage model using
circular transmission range

The coverage hole recovery algorithms either use the static nodes, which
are in sleep mode during the idle phase, or mobile/redundant nodes,
which are deployed in the network as per requirement. Previous studies
have either used the optimization-based node deployment schemes
{[}9{]}{[}10{]} {[}11{]} or reinforcement learning based schemes for
hole recovery {[}3{]}{[}8{]}{[}12{]}. The reinforcement learning (RL)
scheme is more adaptive and intelligent for 6G IoT and can adapt more
decision variables in the objective of redundant nodes deployment. The
scalability is also better handled by the RL. The redundant node
deployment is, although with a primary objective of maximum coverage, it
should also solve the energy consumption and the long lifetime of
neighboring nodes too. The previous research had used the states in RL
as the decision parameters. These are coverage area in {[}3{]}{[}8{]},
reduced hole area and overlapped coverage in {[}12{]}, the cell number,
sensing range, total lifetime of the node {[}13{]}, state of node as
redundancy or isolated based upon the reward of coverage area to sensing
range {[}14{]}. In these previous studies, the action is the location of
the redundant node for maximum coverage area. Although there are a few
other notable characteristics too that are affected by the redundant
node's location and if ignored, may lead to the generation of new
coverage holes in the IoT in a short span of time. These are location
criticality {[}15{]}, load balancing {[}16{]} and failed device priority
value {[}16{]}. Out of these, the first two are dependent upon the
redundant node's location, and the third one depends upon the failed
node's location. In the proposed work in this paper, the RL algorithm
states that variables are a set of unnoticed variables for a longer
network lifetime, and for the first time, this research suggests it
along with minimum energy consumption and maximum coverage area
objectives.

\hypertarget{challenges-contributions}{%
\subsection{Challenges \&
Contributions}\label{challenges-contributions}}

In the context of decision variables, the lack of load balancing,
location criticality for longer lifetime is ignored in the previous
studies, whereas failed device priority is another important parameter
for the quick restoration of services in the emergency areas in 6G IoT.
However, the IoT is a connected network and it involves many spatial and
topological attributes to look for the redundant nodes' deployment.
Spatial features refer to the physical arrangement and distribution of
nodes in the network, while topological features encompass the
interconnections between these nodes and their functional relationships.
Ignoring these critical aspects can lead to suboptimal deployment of
redundant nodes, resulting in inefficient coverage hole recovery
strategies. The lack of a holistic view that integrates both spatial and
topological information may also contribute to the rapid emergence of
new coverage holes, undermining the overall network reliability and
performance. Furthermore, existing reinforcement learning frameworks
often fail to account for interactions between neighboring nodes, which
can significantly influence the effectiveness of coverage strategies. By
incorporating spatial and topological dimensions into the state
representation, RL algorithms could improve adaptability and
responsiveness to dynamic network conditions, ultimately enhancing
coverage reliability and energy efficiency.

To deal with challenges in the redundant nodes placement in the hole
healing, the following is the contribution in this paper:

\begin{itemize}
\item
  Failed device priority, location criticality, load balancing and
  energy consumption will be considered as the primary decision
  variables in the 6G powered IoT network
\item
  The graphical convolution network (GCN) is first time to be introduced
  in this work to extract the spatial and topological attributes from
  the above set of decision variables for the multiple nodes which makes
  it generalize for different network sizes.
\item
  Model free Deep Reinforcement Learning (DRL) is used to select the
  optimal location and number of redundant nodes to be placed in the
  network, considering the graphical spatial-topological attributes from
  the GCN.
\end{itemize}

The novel solution to address coverage hole healing involves two steps:
using a new set of decision variables for placing redundant nodes to
extend network lifetime, and extracting spatio-topological embeddings
from IoT node connections to train DRL for optimal node placement.

\hypertarget{paper-organization}{%
\subsection{1.2. Paper organization}\label{paper-organization}}

\hypertarget{system-model-and-problem-formulation}{%
\section{System Model and Problem
Formulation}\label{system-model-and-problem-formulation}}

\hypertarget{g-iot-network-as-graph}{%
\subsection{2.1. 6G-IoT Network as Graph}\label{g-iot-network-as-graph}}

Extending the work from {[}4{]}, the IoT is presented as an undirected
graph\(\ G = (V,\ E)\). The IoT network in 6G is not a static network as
it is a ubiquitous device connection, and the mobility of devices makes
it dynamic. The solution of coverage hole healing in this proposed
scheme involves the placement of redundant nodes, which are also moving
nodes, either deployed on any vehicle or any robotic sensor node.
Although in the present work, once the optimal location is identified
for the redundant nodes, these become static. The operational
chrematistics of redundant IoT nodes are similar to on-demand RIS in 6G
to provide coverage to places with line-of-sight obstruction to reduce
energy consumption and cost {[}17{]}. So, the network graph will be a
continuously evolving structure in 6G IoT with respect to time.

The set of nodes are vertices \(v \in V\) and the set of edges are
communication links\(\ e \in E.\ E\) is the set of edges \((u,\ v)\)
generates when nodes \(u\) and \(v\) are in the communication range of
each other. The neighbor of each node is defined during the generation
of the graph. A distance matrix \(d(u_{i},v_{i})\) for each node is
created where \(i \in N\ \)is the number of nodes present in the
network. The node is said to be in the neighborhood if the distance is
less than the sensing radius \(R_{s}\). An adjacency matrix \(A\) is
created here for the cell element as 1 for nodes in range.

\(A\left( d(u_{i},v_{i}) \right) = \ \left\{ \begin{matrix}
1,\ \ \ \ if\ \ \ d\left( u_{i},v_{i} \right) \leq \ R_{s},\ i \in N\  \\
0,\ \ \ \ \ \ \ \ \ \ \ \ \ \ \ otherwise \\
\end{matrix} \right.\ \) (1)

This matrix \(A\) creates an IoT-connected graph. The elements entries
with element value 1 are considered to be connected. For example
\(A_{1,2} = 1\), it suggests that vertex \(v_{1}\) and \(v_{2}\) are
connected with an edge length \(e_{1,2} = d_{1,2}\).

\includegraphics[width=0.9\textwidth]{images/image2.png}

Figure 2: Graphical Network connection in 6G IoT with coverage hole and
redundant node location spatial and topological features

The graph construction on the basis of the distance connection only
gives the spatial relationship but lacks the topological and functional
relationships with the other nodes. The location criticality
(\(C_{\text{critical}}\)), load balancing (\(f_{\text{load}}\)) and
energy consumption (\(E_{\text{cons}}\)) are the topological
relationships attributes. The value of these attributes changes with the
simulation time as these are dependent on the neighbouring nodes. A node
\(x_{v_{1}}\epsilon\{{C_{\text{critical}}}_{1},f_{\text{loa}d_{1}},\ {E_{\text{cons}}}_{1}\}\)
is a set of these attributes to generate the topological relationships
with other sensor nodes. Figure 2 shows the graph \(\mathcal{G}\) and
its topological and spatial connections between the vertices. The
embeddings as topological attributes are extracted using Graphical
convolutional network (GCN) which uses a layer propagation rule for
topological features as in equation 2 {[}18{]}:

\(f\left( x_{v_{i}}\left( t \right),\ A_{t} \right) = H_{t}^{\left( l + 1 \right)} = \sigma(D^{- \frac{1}{2}}\ {\widetilde{A}}_{t}\ {x_{v_{i}}\left( t \right)\ \widetilde{D}}^{- \frac{1}{2}}\ H_{t}^{l}\ W_{t}^{l})\)
(2)

Here \(H_{t}^{\left( l + 1 \right)}\) represents the embeddings of the
next layer \(l + 1\) and \(x_{v_{i}}\left( t \right)\) are the features
of the ith sensor node at \(t = 0\). The GCN in {[}18{]} is developed
for the semi-supervised classification problem, though the work in the
current manuscript is not the classification work, yet it is using the
strength of GCN layers' embeddings which considers the spectral
convolution on the IoT network graph with the defined nodes' features
\(x_{v_{i}}\left( t \right)\).

The GCN updates the nodes' features iteratively by aggregating
information from the neighbouring nodes using graph convolution
operators. This process reflects how sensor nodes influence each other's
positions and states within the network over time. This helps in
capturing the interactions between nodes which falls in the category of
topological attributes. The graph convolution operators not only update
the connections but also nodes' features \(x_{v_{i}}\left( t \right)\),
The learned adaptive feature representations are low-dimensional vectors
of nodes at a time, referred to as \(H_{t}^{\left( l + 1 \right)}\) at
the time t and at layer \(l\), which is 2 in this paper. It gets
dynamically updated during the iterations. Figure 3 shows the change in
the coverage area in the network with respect to simulation period. To
validate and visualise the problem formulation, we have clustered the
embeddings in t-SNE plot at few time simulation instant using k-means.
However, in the core of this work, only trained embeddings from the GCN
are used as state variables in the deep reinforcement learning as
described in section 3.2. The figure 3 represents the network topology
change after the random placement of redundant node. The silhouette
score indicates the better coverage after the placement of redundant
nodes.

\includegraphics[width=0.9\textwidth]{images/image3.png}

Figure 3: GCN embeddings t-SNE plot for 6G IoT network without holes,
with different placement of redundant nodes at different positions

The decision variables formulation is discussed in the next sub-section
which is used to calculate the nodes features for embeddings extraction
to be used as state variables in model free redundant nodes placement.

\hypertarget{problem-formulation-of-decision-variables-for-hole-coverage}{%
\subsection{2.2. Problem formulation of Decision Variables for Hole
coverage}\label{problem-formulation-of-decision-variables-for-hole-coverage}}

The decision variables in the placement of redundant nodes are location
criticality, load balancing, energy consumption which are directly
related to location of the redundant node and failed device priority as
the fourth one which is calculated from the location of the failed node.
These are described as following:

As is validated from figure 3, the coverage area changes with the
deployment of the redundant node and the information processing load
also varies which may lead to non-uniform energy consumption and
generation of coverage hole into another area after few simulation
period {[}16{]}. The failure of the new node forces the neighbouring
node to follow the longer path for data transmission which results in
the probability of new coverage hole generation. Hence, the load
balancing consideration in the redundant node's deployment is important.

To address this problem, it is to assume that the balanced clusters can
have the uniform residual energy distribution and number of member
nodes. So, the optimal position of the redundant node is considered with
uniform load on the cluster head. Additionally, nodes' proximity can
also be considered to ensure less energy consumption. Where
\(f_{\text{load}}\) is defined as the effective fitness function for
load balancing of cluster head and can be calculated using equation
below:

\(f_{\text{load}} = \left( 1 - \frac{\mu_{\text{load}}}{\text{CH}_{\max}} \right) + \left( \frac{\text{loaded\ cluster\ head}}{\text{total\ cluster\ head}} \right)\)
(3)

Where, \(\mu_{\text{load}}\) defines the mean of load can be calculated
by equation 4, \(\text{CH}_{\max}\) is the CH with maximum load.

\(\mu_{\text{load}} = \frac{\sum_{i = 1}^{m}{\text{Load\ }\left( \text{CH}_{q} \right)}}{\text{cluster\ heads}}\)
(4)

\(\text{Load\ }\left( \text{CH}_{q} \right) = \ N \times \frac{E_{\text{remain}}\ \left( \text{CH}_{q} \right)}{E_{\text{initial}}\ \left( \text{CH}_{q} \right)}\)
(5)

Where, \(q\) is the number of CH and \(N\) denotes number of sensor
nodes, \(E_{\text{initial}}\ \left( \text{CH}_{q} \right)\) and
\(E_{\text{remain}}\ \left( \text{CH}_{q} \right)\) are the initial and
remaining energy of cluster head \(\text{CH}_{q}\) respectively. The
load on the cluster head increases with the number of sensor nodes and
packet size. Therefore, the remaining energy is calculated by
subtracting the consumed energy given in equation 6 from the initial
energy given as follows {[}19{]}{[}24{]}:

\(E_{\text{cons}}\left( N,d \right) = \begin{Bmatrix}
N \times E_{\text{elec}} + N \times \in_{\text{fs}} \times d^{2},\ \ \ \ \ \ \ \ \ \ \ \ \ \ d < d_{0} \\
N \times E_{\text{elec}} + N \times \in_{\text{mp}} \times d^{4},\ \ \ \ \ \ \ \ \ \ d \geq d_{0} \\
\end{Bmatrix}\) (6)

Where, \(d_{0}\) defines the distance between sender and receiver, N
denotes the nodes, and \(E_{\text{elec}}\) is the energy required by
electronic circuitry. \(\in_{\text{mp}}\) , is Multipath channel energy
and \(\in_{\text{fs}}\), is free space energy.

The failed device priority is also location and energy dependent. The
work in {[}16{]} calculates the failed device priority on both, however,
in this current work, the energy consumption is to be involved in reward
calculation in the DRL, so the priority is calculated solely on the
basis of neighbourhood degree. It is calculated as follows:

\(\phi_{i} = \sum A\left( d(u_{i},v_{i}) \right)\text{\ \ }\) (7)

Higher the \(\phi_{i}\), more is the priority of the failed device. The
location criticality (\(C_{\text{critical}}\)) for the redundant node is
also calculated based on the neighbourhood degree with adaptive graph
construction with every new position of the redundant node. The equation
7 is modified and calculated for the redundant node's neighbourhood
degree.

These network topological features are fed into the GCN to extract the
spatio-topological embeddings to be used in the DRL as discussed in the
next section of the complete proposed solution of coverage hole healing.
The next section of the article discusses the proposed solution using
these embeddings into DRL for the hole coverage.

\hypertarget{proposed-solution}{%
\section{3. Proposed Solution}\label{proposed-solution}}

\hypertarget{overview}{%
\subsection{3.1 Overview}\label{overview}}

Previously in this paper, the 6G IoT network is presented as network
graph to extract the spatio-topological features. The decision variables
are discussed in section 2.2. these decision variables are objectives of
the hole recovery. With the placement of redundant nodes, the maximum
hole area (\(A_{\text{hole}}\)) should be covered with lesser number of
redundant nodes (\(N_{r}\)) with the consideration of the minimization
of the \(E_{\text{cons}}\). The objective function of the proposed
methodology can be defined as in equation 8:

\(f_{\text{obj}} = \ argmin\left( {w_{1} \times N}_{r} + w_{2} \times \sum_{i \in \left( 1,N \right)}^{}A_{hole,i} + w_{3} \times \sum_{i \in \left( 1,N \right)}^{}E_{\text{cons}} \right)\)
(8)

Here \(w_{1},w_{2}\), \(w_{3}\) are random weights \(\in (0,1)\) with
condition of (\(w_{1} + w_{2} +\) \(w_{3}\  = 1)\). Here the
maximization of coverage area is termed as minimization of the hole area
(\(A_{hole,i}\)) to bring the objective function uniformity as rest
other two contributing functions tend to be minimized. With the
objective of equation 8, the graphical deep reinforcement learning
(GDRL) has been proposed to place the \(N_{r}\) optimally in the
network. It is termed as GDR now as the saptio-topological features from
the Graph convolutional network (GCN) are extracted using the decision
variables in equations 3,6,7 and these are to be used by DRL to generate
the optimal action space. This methodology is also termed as model free
approach as it doesn't require the understanding of the environment and
takes the action as per the change and feedback from the 6G IoT network
environment {[}20{]}. The overview of the complete methodology is shown
in figure 4. The set of decision variables \(x_{v_{i}}\left( t \right)\)
from the network are fed into the GCN and trained for every iteration of
DRL. The new embeddings thus generated are input to the DRL and
corresponding action by the agent is decided as the redundant nodes'
positions
\(\left\lbrack x_{j},y_{j} \right\rbrack,\ j\epsilon 1..N_{r}\). The
reward is calculated on the measured node's energy consumption in few
network simulations and a new policy of the agent decides the new action
as in figure 4.

\includegraphics[width=0.9\textwidth]{images/image4.png}

Figure 4: Methodology Diagram of the proposed spatio-topological
embeddings consideration for Redundant nodes placement by DRL

The DRL is based upon the Markov decision process with 5 tuple
(\(S,U,P,\gamma,R\)), where \(S\) is the state space variables which is
the feedback from the environment, \(U\) is the set of actions of
environment variable to be tuned as input to change the states, \(P\) is
the transition of one state \(s\) to new state \(s^{'}\) at next time
interval using action \(A\), \(R\) is the reward which motivates or
penalize the agent for the right direction learning. \(\gamma\) is the
reward discount parameter which is usually kept fixed at 0.95 and
considered as the sensitive variable {[}21{]}. Skipping the more on MDP
formulation, we suggest reader to explore more in {[}21{]}.

Various DRL networks are existing in the literature and the selection of
those is defined by the type and size of action space. The action space
in the proposed work is continuous as it's the location of redundant
node. Out of policy based, value based and hybrid types of DRL
methodologies, this work has opted the hybrid DRL as for large
continuous space, first two struggles with the convergence {[}22{]}. The
deep deterministic policy gradient (DDPG) is used as DRL algorithm in
this work. As discussed in section 2, the state space is a set of
embeddings for every node
\(S_{i\left\{ 1,..N \right\}}(t) = \{ H_{1\ }^{l + 1}\left( t \right),\ H_{2\ }^{l + 1}\left( t \right)\ldots.H_{\text{n\ }}^{l + 1}\left( t \right)\}\)
which is continues in nature and size of each embedding depend upon the
\(l + 1\) layer as discussed in section 3.2. The action
\(u\left( t \right)\) is the set of \(N_{r}\) and location coordinates
\(\left\lbrack x_{j},y_{j} \right\rbrack,\ j\epsilon 1..N_{r}\). Actions
are primary outcomes of the agent in the DRL and impacts the environment
to generate the new state \(s(t)\). The transition probability is thus
solely connected to action \(u(t - 1)\) and state \(s(t - 1)\) for new
\(u(t)\) and \(s\left( t \right)\) as:

\(P\ (s^{*};s,\ u\mathbb{)\  = \ P\{}s\ (t)\  = \ s^{*}\ |s\ (t - 1)\  = \ s,\ u(t - 1)\  = \ u)\},\ \ \ s,s^{*}\mathcal{\in S,\ }u \in \ U\)
(9)

The reward of the GDRL is computed as the extension of equation 8. The
learning of the agent is done to maximize the reward \(r_{t}\). The
total reward at time \(t\) with state \(s(t)\) is
\(R\left( \text{s\ }\left( t \right),\ t \right) = \sum_{k = 0}^{M - t}\gamma^{k}\text{\ r\ }\left( t + k \right).\)
The attenuation coefficient has the value range of
\(\gamma\epsilon(0,1\rbrack\). The reward in the transition from state
\(s(t + k - 1)\) to \(s(t + k)\) is \(r(t + k)\). The agent trains the
best policy \(\pi^{*}\) to maximize the incentivization of the agent
i.e. \(\max{R(t)}\).

The objective of the \(N_{r}\) placement is to maximize the coverage
rate \(C_{r}\) after the nodes placement. The cost of the reward is
positive if the \(C_{r} > \tau\) else agent is penalized. The \(\tau\)
is the threshold for the coverage rate of the hole area. It is
calculated as in equation 10 {[}13{]}.

\(C_{r} = \frac{\sum_{i = 1}^{A_{\text{hole}}}{\pi R_{s}^{2}}}{A_{s}}\)
(10)

Here \(A_{s}\) is the total sensing area and the numerator denotes the
total covered sensing area with sensing radius \(R_{s}\). The cost thus
calculated is impacted by the \(\mathrm{\Delta}C_{r} = C_{r} - \tau\)
as:

\(\text{Cos}t_{c_{r}} = \ \left\{ \begin{matrix}
e^{\mathrm{\Delta}C_{r}}\ \ \ \ \forall\ \mathrm{\Delta}C_{r} > 0 \\
 - e^{\mathrm{\Delta}C_{r}}\ \ \ \forall\mathrm{\Delta}C_{r} < 0 \\
\end{matrix} \right.\ \) (11)

The second objective of the nodes placement is the minimization of the
number of redundant nodes which is improved for the incentivization of
the DDPG agent. The network is divided into clusters based on the Eigen
Heuristics method {[}23{]}, and the uniform load distribution on
\(CH_{q}\) is considered as the positive or negative cost calculation
for the redundant nodes' placement. The normalized
\(f_{\text{load}}\ \)from equation 3 is termed as second objective in
the reward calculation. Similarly, the remaining energy maximization
\(E_{\text{rem}} = 1 - E_{\text{cons}}\) from equation 6 is used as the
maximization of reward. The reward at any iteration is calculated as in
equation 12

\(r_{t} = \alpha_{1}\text{Cos}t_{c_{r}} + \alpha_{2}f_{\text{load}} + \alpha_{3}E_{\text{rem}}\)
(12)

The \(r_{t}\) is calculated in every iteration of the agent training and
accordingly agent takes the decision for the \(u_{t + 1}\ \) update.

The pseudocode of the proposed solution is given in algorithm 1. The
present methodology aims to locate the optimal placement of the
redundant nodes (\(x_{r},y_{r}\)) using spatio-topological features
using GDRL. The hybrid of the value network \(Q(s,\ u|\theta_{Q})\) and
policy network \(\mu(s|\theta_{\mu})\) in the DDPG works together to
derive the optimal location of the node. The input to the algorithm 1 is
the coverage hole and network parameters whereas output of it is the
position of redundant nodes and volume of them. After a few simulation
seconds run of the network, the set of topological features are updated.
The two layered GCN as in equation 2 extracts the spatio-topological
features considering all four decision variables discussed in section
2.2. Step 12 in algorithm 1 defines the stopping criteria of the GDRL.
If the condition of overlapping of two redundant nodes is violated, the
\(a_{t}\) at that iteration of the agent is discarded and reset.

Algorithm 1: Pseudocode for the proposed Spatio-topological extraction
with GDRL optimized redundant nodes placement

\begin{longtable}[]{@{}
  >{\raggedright\arraybackslash}p{(\columnwidth - 0\tabcolsep) * \real{1.00}}@{}}
\emph{\textbf{Input:}} \(N_{\text{hole}}\), \(A_{\text{hole}}\),
\(R_{s}\), \(N\), \(R_{c}\) \\
\emph{\textbf{Output:}}\(\ x_{r}^{j = 1..N_{\text{hole}}},\ y_{r}^{j = 1..N_{\text{hole}}}\)\emph{,}
\(N_{r}\) \\
 \\
\end{longtable}

Assuming the hole locations and hole area \(A_{\text{hole}}\) with
number of holes \(N_{\text{holes}}\) are already identified in the hole
detection phase (out of context of this paper, as is published in
{[}4{]}). The next section focuses on the layered architecture of the
DDPG.

\hypertarget{layered-architecture-of-ddpg-actor-and-critic}{%
\subsection{3.2 Layered Architecture of DDPG Actor and
Critic}\label{layered-architecture-of-ddpg-actor-and-critic}}

The proposed DDPG model comprises an actor network and a critic network,
which form its architecture. The actor network determines the redundant
nodes' positions, known as the action \(a_{t}\). The critic network
produces the action-value \(Q^{\pi}(s,a)\).

\begin{enumerate}
\def\labelenumi{\roman{enumi}.}
\item
  \emph{Actor Network}
\end{enumerate}

It is important to identify valuable characteristics from the input
data, specifically the states of the others nodes when a new redundant
node is placed in hole area. The actor-network is responsible for
selecting the optimal course of action based on a specific situation.
The actor network consists of an input layer, fully connected layers,
and activation functions. The objective of this activity is to optimize
the anticipated outcome (reward) by engagement with the environment,
namely an 6G IoT network. The actor network architecture can be seen in
Figure 5. The network begins with a feature input layer designated as
"state" to represent the current state. The dimensions are
\(3 \times 1\), denoting three distinct features that represent aspects
of the state as given in section 3.1. Then, the first fully connected
layer processes the input layer with weights \(400 \times 3\) and bias
\(400 \times 1\). ReLU activations are utilized subsequent to the first
fully connected layer in order to induce non-linearity, as can be seen
in Figure 5. ReLU facilitates the acquisition of intricate patterns by
implementing a straightforward yet efficient non-linear transformation.
Again, a second fully connected layer with \(300 \times 3\) activations,
further processing the output of the previous layer. The learnable size
is \(300 \times 400\) weights and \(300 \times 1\) biases. Lastly, the
third fully connected layer following the ReLU layer has the size of
\(7 \times 1\). A hyperbolic tangent activation function is used after
this third layer. This function squashes the output between -1 and 1,
which can be useful for actions that need to be within a specific range.
At the end, A scaling layer is applied to adjust the range of the output
actions. This can be critical for normalizing actions to the scale
expected by the energy management environment.

Table 1. Description of different layers of Actor network

\begin{longtable}[]{@{}
  >{\raggedright\arraybackslash}p{(\columnwidth - 6\tabcolsep) * \real{0.25}}
  >{\raggedright\arraybackslash}p{(\columnwidth - 6\tabcolsep) * \real{0.25}}
  >{\raggedright\arraybackslash}p{(\columnwidth - 6\tabcolsep) * \real{0.25}}
  >{\raggedright\arraybackslash}p{(\columnwidth - 6\tabcolsep) * \real{0.25}}@{}}
Sizes} \\
State (three features) & Input layer & \(3 \times 1\) & - \\
FC\textsubscript{1} (400 fully connected layer) & fully connected &
\(400 \times 1\) & weights \(400 \times 3\)
bias \(400 \times 1\) \\
ReLU1 & ReLU layer & \(400 \times 1\) & - \\
FC\textsubscript{2} (300 fully connected layer) & fully connected layer
& \(300 \times 1\) & Weights \(300 \times 400\)
Bias \(300 \times 1\) \\
ReLU2 & ReLU layer & \(300 \times 1\) & \\
FC\textsubscript{3} (7 fully connected layer) & fully connected layer &
\(7 \times 1\) & Weights\(\ \ 7 \times 300\)
Bias \(7 \times 1\) \\
Tanh1 & hyperbolic tangent & \(7 \times 1\) & - \\
Scale & scaling layer & \(7 \times 1\) & - \\
\end{longtable}

\begin{enumerate}
\def\labelenumi{\roman{enumi}.}
\setcounter{enumi}{1}
\item
  \emph{Critic network}
\end{enumerate}

The objective of this study is to train the critic network to accurately
predict the Q-values so that the actor network can learn to make optimal
decisions. The critic network assesses the efficacy of the activities
proposed by the actor-network by considering the reward, which is
determined by factors as in equation 12. The critic network structure
comprises several input layers and fully connected layers tailored to
process state and action inputs, ultimately predicting a Q-value. The
input layers consist of a State Input layer, which has dimensions of
\(3 \times 1\) and represents the current state of the energy system,
including variables like power loss, cost, and power deficit as given in
section 3.1. Additionally, there is an Action Input layer with
dimensions of \(7 \times 1\), capturing the actions taken by the actor
network, such as decisions regarding \({(x}_{r},\ y_{r})\). The fully
connected layers begin with FC\textsubscript{1}, which features
\(400 \times 1\) activations, processing the state input with a size of
\(400 \times (3 + 1\)). The weights and biases can be seen table 3. The
next layer, FC\textsubscript{2}, also has \(400(C) \times 1(B)\)
activations and integrates processed state and action information. Its
size is \(400 \times (400 + 7),\) indicating it combines features from
both the state and action. The final fully connected layer,
FC\textsubscript{3}, has 1(C)×1(B) activation, reducing the output to a
single value that represents the Q-value. The critic network
architecture can be seen in Figure 5. To introduce non-linearity as we
discussed in above given actor network, ReLU layers are used as
activation functions between layers, known for their simplicity and
efficiency in handling gradients. An addition layer sums the inputs,
likely combining state and action inputs after their respective
processing through earlier layers.

Table 2. Description of different layers of Critic network

\begin{longtable}[]{@{}
  >{\raggedright\arraybackslash}p{(\columnwidth - 6\tabcolsep) * \real{0.25}}
  >{\raggedright\arraybackslash}p{(\columnwidth - 6\tabcolsep) * \real{0.25}}
  >{\raggedright\arraybackslash}p{(\columnwidth - 6\tabcolsep) * \real{0.25}}
  >{\raggedright\arraybackslash}p{(\columnwidth - 6\tabcolsep) * \real{0.25}}@{}}
Sizes} \\
State (three features) & Input layer & \(3 \times 1\) & - \\
FC\textsubscript{1} (400 fully connected layer) & fully connected &
\(400 \times 1\) & weights \(400 \times 3\)
bias \(400 \times 1\) \\
ReLU1 & ReLU layer & \(400 \times 1\) & - \\
Action (7 features) & Features input & \(7 \times 1\) & - \\
Add (element-wise addition of 2 inputs) & Addition & \(400 \times 1\)
& \\
FC\textsubscript{2} (400 fully connected layer) & fully connected layer
& \(400 \times 1\) & Weights 4\(00 \times 7\)
Bias 4\(00 \times 1\) \\
ReLU2 & ReLU layer & \(400 \times 1\) & \\
FC\textsubscript{3} (7 fully connected layer) & fully connected layer &
\(1 \times 1\) & Weights\(\ \ 1 \times 400\)
Bias 1\(\times 1\) \\
\end{longtable}

\includegraphics[width=0.9\textwidth]{images/image5.png}

Figure 5. The structure of actor critic network in DDPG for 6G IoT hole
healing

\hypertarget{results-discussion}{%
\section{Results \& Discussion}\label{results-discussion}}

The proposed algorithm starts with the random deployment of sensor nodes
in 6G IoT area. The validation of the proposed algorithm is done at
different \(R_{s}\) and network area. It is assumed that the coverage
area and location is already identified using {[}4{]}. The primary
objective to maximise the coverage area with least energy consumption
and uniform loading factor. The directed placement of \(N_{r}\) by
spatio-topological features learning change the network topology and new
clusters generation is done by GCN following the section 2.1. The
residual energy and loading factor are calculated on the CH after the
100 simulation rounds. The implementation of the proposed work is done
in MATLAB as well as pytorch. The node's features are generated in
MATLAB and fed into pytorch module of GCN for the embeddings
\(H_{t}^{\left( l + 1 \right)}\) generations. The DDPG agent is trained
using MATLAB toolbox. Whenever a new action \(u_{t + 1}\) for a state
\(s_{t}\) is generated by the trained agent policy
\(\mu\left( s \middle| \theta^{\mu} \right)\), the new embeddings are
calculated and new CH features \(x_{q\epsilon 1,2..N_{q}}\) are used for
reward calculation form equation 12. In the previous work, the hole
healing problems has been dealt with either by heuristic optimization
algorithms {[}9{]}{[}11{]} or with reinforcement learning
{[}3{]}{[}8{]}{[}13{]}. The distinction between the proposed work and
the state-of-the-art is the use of the topological features embedding
from the GCN followed by DDPG agent training for model free approach.
The uniform loading after the \(N_{r}\) placements constrained by the
overlapping area of redundant nodes make the present work unique and
novel.

Since the failed device priority and location criticality are related
terminologies and dependent upon the number of neighbouring connections,
the embeddings as the input to the GCN are
\(E_{\text{cons}},\ f_{\text{load}}\) and \(\phi_{i}\). Further in the
discussion of the results, the validation and difference between the
hole healing performance with maximization of equation 12 is shown in
figure 6. The performance improvement in consideration of
spatio-topological features as a novel contribution is justified in
figure 6 itself. The use of GCN to extract the relationship of the nodes
with other in the network facilitates the decision making more reliable
and accurate in the redundant node placement, whereas the common DDPG
with same hyperparameters a in table 1 is used for the \(u_{t}\). The 6G
IoT network is simulated for 100 rounds. The optimal number of clusters
calculation following the step 2 for 100 nodes in the network comes to
be \(q = 5\). It has been noticed in various iterations with different
nodes density and respective \(x_{v_{i}},\) the optml number of clusters
comes to be approximate \(q = 5\). So, to reduce the steps in the
methodology, it is now fixed for different nodes density in the network.
The use of spatial features and spatio-topological features are
separately trained with DDPG for a network size of
\(200 \times 200m^{2}\) and with \(R_{s} = 30\ m\). The bar chart shown
in figure 6 indicates the effect of graphical connections between the
nodes in hole healing.

\includegraphics[width=0.9\textwidth]{images/image6.png}

Figure 6: Comparison of spatial and spatio-topological features impact
on the hole coverage performance

Table 1. Training hyperparameters of 6G IoT DDPG network

\begin{longtable}[]{@{}ll@{}}
Sample Time & 1 \\
Target Smooth Factor & \(10^{- 3}\) \\
Discount Factor & 0.99 \\
Mini Batch Size & 64 \\
Experience Buffer Length & \(10^{- 6}\) \\
Actor learning rate & 0.01 \\
Critic learning rate & 0.01 \\
\end{longtable}

The proposed algorithm is evaluated on different number of network size,
transmission range and number of holes. The state-of-the-art works as
discussed earlier have used either reinforcement leaning and heuristic
optimization algorithms. In the work of {[}3{]}{[}8{]}{[}13{]}, the
action variables are redundant nodes' location however these differ in
the \(s_{t}\) and \(r_{t}\) definition. In {[}3{]}, the reward is being
calculated with maximum coverage area and minimum energy consumption in
the mobile redundant node. Though, this work has ignored the overall
performance of the network which is defined by the energy residual of
the relay node (cluster head) and dependent on the uniform load
distribution. The discrete action space makes it more prone to fall in
local minima as q-learning has been used in the work. Q-learning defines
a discrete action space within a Q-table, which limits its adaptability
to the dynamic network topology of 6G. The objective of the hole healing
in {[}8{]} is to maximize the working time of the covered subject which
is again possible by minimum energy consumption and fulfilled by the
q-learning itself. The work in {[}14{]} is targeting to maximize the
coverage area with minimum overlapping area. These state-of-the-art
works are used for the comparison of suggested work in this manuscript
as these make the founding ground of selection of spatial features. The
feature gap to increase the network performance after placement of
\(N_{r}\) is filled with the novel combination \(x_{t}\) and topological
characteristics using \(x_{t}\).

The comparison with the {[}3{]}{[}8{]}{[}13{]} is shown qualitatively in
this work as the objective of these state-of-the-art works are
collectively used in our work and simulated with the q-learning too.
Figure 7 shows a comparison of the proposed scheme with the q-learning
as it has been used in all research considering discrete action space
and spatial features combination instead of using them individually with
q-learning. Due to difference in simulation environments, the combined
spatial features' set simulation with q-learning and simulation with
spatio-topological DRL simulation, gives the flexibility to validate the
hypothesis of novel features set performance with DDPG. For 100
simulation time stamps with randomly deployed IoT sensors, the proposed
scheme's
\(x_{r}^{j = 1..N_{\text{hole}}},\ y_{r}^{j = 1..N_{\text{hole}}}\) is
giving the highest \(E_{\text{residual}}\) over different number of
holes in the area.

\includegraphics[width=0.9\textwidth]{images/image7.png}

Figure 7: Performance Comparison of Proposed methodology with the
spatial features with DRL placement, and Q-learning as the training
agent with both spatial and spatio-topological features with varying
number of holes

In the figure 7 analysis with varying number of holes, the reward
\(r_{t}\) in equation 12 is maximized. The residual energy in the
proposed spatio-topological features and DDPG training is highest,
however due to probabilistic RL and random deployment of the sensors,
the improvement is non-uniform with varying number of holes. The
second-best residual energy is achieved when the same proposed features
set are trained by q-learning considering the discrete action space
limiting the exploration of the search area by the agent. The maximum
improvement of \(E_{\text{res}}\) using proposed scheme is 13.1\% and
minimum is 0.8\%. The \(E_{\text{res}}\) is not monotonically decreasing
due to different spatial distribution of holes and constant \(N_{r}\)
irrespective of the number of holes. The \(C_{\text{critical}}\) and
\(\phi\) are correlated defined by equation 7. This equation is the sum
of all nodes in the sensing range or in other terms, it is giving the
number of neighbours. Higher the number of neighbours, better the
location to place the recovery node. The subplot 2 in the figure 7 shows
the reduction in neighbourhood degree with the increase in holes in the
network; however, no single approach is showing the consistency and
variance is also very less, so the range of the change in the
neighbouring connections w.r.t. holes is observed by the envelope and
mean line indicates the expected behaviour. On the contrary of this
parameter performance, the \(f_{\text{load}}\) has shown the significant
improvement in the proposed spatio-topological features with DDPG
training, than the q training. It has been validated earlier in figure 6
too that the spatial features only are less performing than the
spatio-topological features, so is validated again in figure 7 too. A
maximum improvement of \(f_{\text{load}}\) in figure 7 is 19.4\% and
minimum is 10.62\%. This significant loading factor improvement is
achieved with the topological characteristics of nodes' spatial features
with GCN.

This behaviour of the methods used in comparison can be better explained
within the context of our multi-objective framework. As demonstrated in
the residual energy and load balancing comparisons, the proposed method
achieves superior performance in these metrics while maintaining
competitive connectivity. The envelope representation (dashed lines)
captures the inherent variability due to topological transformations as
hole density increases. Particularly noteworthy is proposed method's
ability to maintain connectivity comparable to Spatio-temporal in
q-learning' while delivering 13.1\% better energy efficiency and 19.4\%
improved load balancing at moderate hole densities (15-25 holes). This
represents a favourable trade-off in practical WSN deployments where
network longevity is often prioritized over maximum connectivity.

The simulation of figure 7 is done with 100 nodes distributed in the
area of \(200 \times 200m^{2}\). The smaller search area gives the
advantage to discrete action space too; however, the proposed model free
training advantage is clearly validated in figure 8 in higher
distributed area with varying number of sensor density.

\includegraphics[width=0.9\textwidth]{images/image8.png}

Figure 8: Performance comparison of proposed methodology with the q
learning and spatial features trained on DDPG and Q learning for varying
sensor density

The analysis is done considering the number of holes where the results
in figure 7 are under performing. Figure 8 keeps the
\(N_{\text{hole}} = 40\) for varying sensor densities. The
\(E_{\text{res}}\) is increasing with the increase in number of nodes
due to stable connectivity as validated from the neighbouring curve. The
neighbours of the cluster heads are stable due to spatio-topological
features. It indicates the uniform distribution of load too. The
\(f_{\text{load}}\) is increasing due to equation 5 which is using
energy in the calculation. The energy residual has increased with number
of nodes due to fixed deployment area. With the increase in node
density, the distance between nodes reduces and hence energy consumption
as in equation 6. The proposed scheme has shown the nonlinear
improvement from 200-400 nodes' densities and this range can be
considered as the optimal density for the stable connectivity and
performance in 6G IoT. The shortcoming of the proposed
spatio-topological features with DDPG placement of redundant nodes is
validated from figure 8 on a larger scale with 40 failed nodes in the
network creating the connectivity holes.

\hypertarget{conclusion}{%
\section{4. Conclusion}\label{conclusion}}

The coverage hole healing in 6G IoT is proposed in this work. A novel
approach using spatio-topological features have been proposed to make
the dependent decision on neighbouring sensors to the redundant nodes.
The continuous action space as the location of redundant nodes in the
DDPG training algorithm is defined which makes it model free and
adaptive to higher sensor nodes' density too. Topological
characteristics of GCN have been used as state space variables. The
advantage of the proposed methodology is validated by simulating under
various environmental conditions of varying number of holes and varying
nodes density. A significant improvement of 13.1\% in the proposed
scheme than the state-of-the-art q-learning scheme for the holes density
of 5 is achieved and 0.81\% in the residual energy for the higher hole
density of 35 holes in the network. Similarly, the 19.4\% improvement in
the load balancing is noticed with stable connectivity. Furthermore, the
simulation experiment with varying sensor density for 40 holes have
shown the similar behaviour of performance improvements. Considering the
comparison of proposed saptio-topological features and spatial features,
the location criticality also shown an improvement of 16.7\%.

Although the significant improvement with the usage of DDPG in
spatio-topological scenario has been observed yet there is still scope
for the improvement for 6G scalability. Present work has shown the
stable performance between the 200-400 nodes density which can be
improved if topological features with more than single neighbour is
considered. Furthermore, the multiagent DRL can also be helpful in
handling higher hole area in the network.

\hypertarget{references}{%
\section{References}\label{references}}

{[}1{]} Lee, Howon, Byungju Lee, Heecheol Yang, Junghyun Kim, Seungnyun
Kim, Wonjae Shin, Byonghyo Shim, and H. Vincent Poor. "Towards 6G
hyper-connectivity: Vision, challenges, and key enabling technologies."
Journal of Communications and Networks 25, no. 3 (2023): 344-354.

{[}2{]} Cui, Qimei, Xiaohu You, Ni Wei, Guoshun Nan, Xuefei Zhang,
Jianhua Zhang, Xinchen Lyu et al. "Overview of AI and communication for
6G network: fundamentals, challenges, and future research
opportunities." Science China Information Sciences 68, no. 7 (2025):
171301.

{[}3{]} Xia, Yunzhi, Xianjun Deng, Lingzhi Yi, Laurence T. Yang, Xiao
Tang, Chenlu Zhu, and Zhongping Tian. "AI-driven and MEC-empowered
confident information coverage hole recovery in 6G-enabled IoT." IEEE
Transactions on Network Science and Engineering 10, no. 3 (2022):
1256-1269.

{[}4{]}. Gupta, Abhishek, Somesh Kumar, and Manisha Pattanaik. "Coverage
hole detection using social spider optimized Gaussian Mixture
Model."~Journal of King Saud University-Computer and Information
Sciences~34, no. 10 (2022): 9814-9821.

{[}5{]}. Masoud, Mohammad Z., Yousef Jaradat, Ismael Jannoud, and
Mustafa A. Al Sibahee. "A hybrid clustering routing protocol based on
machine learning and graph theory for energy conservation and hole
detection in wireless sensor network." International Journal of
Distributed Sensor Networks 15, no. 6 (2019): 1550147719858231.

{[}6{]}. Koriem, Samir M., and Mohamed A. Bayoumi. "Detecting and
measuring holes in wireless sensor network." Journal of King Saud
University-Computer and Information Sciences 32, no. 8 (2020): 909-916.

{[}7{]}. Beghdad, Rachid, and Amar Lamraoui. "Boundary and holes
recognition in wireless sensor networks." Journal of Innovation in
Digital Ecosystems 3, no. 1 (2016): 1-14.

{[}8{]}. Tang, Yong, Xianjun Deng, Lingzhi Yi, Yunzhi Xia, Laurence T.
Yang, and Xiao Tang. "Collaborative intelligent confident information
coverage node sleep scheduling for 6G-empowered green IoT." IEEE
Transactions on Green Communications and Networking 7, no. 2 (2022):
1066-1077.

{[}9{]}. Hallafi, Ali, Ali Barati, and Hamid Barati. "A distributed
energy-efficient coverage holes detection and recovery method in
wireless sensor networks using the grasshopper optimization algorithm."
Journal of Ambient Intelligence and Humanized Computing 14, no. 10
(2023): 13697-13711.

{[}10{]}. Ma, Donghui, and Qianqian Duan. "A hybrid-strategy-improved
butterfly optimization algorithm applied to the node coverage problem of
wireless sensor networks." Math. Biosci. Eng 19, no. 4 (2022):
3928-3952.

{[}11{]}. Chen, Xinyi, Mengjian Zhang, Ming Yang, and Deguang Wang.
"NHBBWO: A novel hybrid butterfly-beluga whale optimization algorithm
with the dynamic strategy for WSN coverage optimization." Peer-to-Peer
Networking and Applications 18, no. 2 (2025): 1-25.

{[}12{]}. Hajjej, Faten, Monia Hamdi, Ridha Ejbali, and Mourad Zaied. "A
distributed coverage hole recovery approach based on reinforcement
learning for Wireless Sensor Networks." Ad Hoc Networks 101 (2020):
102082.

{[}13{]}. Chauhan, Nilanshi, Piyush Rawat, and Siddhartha Chauhan.
"Reinforcement learning-based technique to restore coverage holes with
minimal coverage overlap in wireless sensor networks." Arabian Journal
for Science and Engineering 47, no. 8 (2022): 10847-10863.

{[}14{]}. Sharma, Anamika, and Siddhartha Chauhan. "A distributed
reinforcement learning based sensor node scheduling algorithm for
coverage and connectivity maintenance in wireless sensor network."
Wireless Networks 26, no. 6 (2020): 4411-4429.

{[}15{]}. Gou, Pingzhang, Gang Mao, Fen Zhang, and Xiangdong Jia.
"Reconstruction of coverage hole model and cooperative repair
optimization algorithm in heterogeneous wireless sensor networks."
Computer Communications 153 (2020): 614-625.

{[}16{]}. Yang, Chen, Shengchao Su, Xiang Ju, and Jiayan Song. "A mobile
sensors dispatch scheme based on improved SOM algorithm for coverage
hole healing." IEEE Sensors Journal 21, no. 18 (2021): 21080-21089.

{[}17{]}. Ndjiongue, Alain R., Octavia A. Dobre, and Hyundong Shin.
"On-Demand RIS-Assisted Free Space Optical Access System for 6 G
Networks." IEEE Transactions on Vehicular Technology (2025).

{[}18{]}. Jiang, Bo, Ziyan Zhang, Doudou Lin, Jin Tang, and Bin Luo.
"Semi-supervised learning with graph learning-convolutional networks."
In Proceedings of the IEEE/CVF conference on computer vision and pattern
recognition, pp. 11313-11320. 2019.

{[}19{]}. Mehta, Shalu, and Amita Malik. "A Swarm Intelligence Based
Coverage Hole Healing Approach for Wireless Sensor Networks." EAI
Endorsed Transactions on Scalable Information Systems 7, no. 26 (2020).

{[}20{]}. Huang, Liping, Jianbin Zheng, Yifan Gao, Qiuzhi Song, and Yali
Liu. "A Lower Limb Exoskeleton Adaptive Control Method Based on
Model-free Reinforcement Learning and Improved Dynamic Movement
Primitives." Journal of Intelligent \& Robotic Systems 111, no. 1
(2025): 24.

{[}21{]}. Maan, Ujjawal, and Yogesh Chaba. "Deep Q-network based fog
node offloading strategy for 5 G vehicular Adhoc Network." Ad Hoc
Networks 120 (2021): 102565.

{[}22{]}. Saleem, Rabbia, Wei Ni, Muhammad Ikram, and Abbas Jamalipour.
"Deep-reinforcement-learning-driven secrecy design for
intelligent-reflecting-surface-based 6G-IoT networks." IEEE Internet of
Things Journal 10, no. 10 (2022): 8812-8824.

{[}23{]}. Kumari, Ashish, Shailender Kumar, and Ram Shringar Raw.
"Modified clustering and incentivized stable CH selection for reliable
VANET communication."~\emph{Cluster Computing}~27, no. 9 (2024):
11983-12005.

{[}24{]}. Fofana, Adama Hawa Rahima, and Jianjun Lei. "Reinforcement
Learning Based Cluster Formation Algorithm in Wireless Sensor Networks."
In Proceedings of the 2nd International Conference on Signal Processing,
Computer Networks and Communications, pp. 367-371. 2023.

\end{document}
